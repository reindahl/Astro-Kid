\section{World Simulaion}
	Due to the complete set of rules in the Astro Kid world not being be described. The rules are constructed by watching the tutorials and playing the game. This means there most likely are some few inconsistency compared to Astro Kid game itself.

	The simulation and editor is created using test driven development and design to follow the paradigm Model View Control 

	\subsection{backend of the simulation}
	All objects in the world is represented as an object with a set of coordinates defining its position in the 2d grid. Multiple objects can have same position, with the exception of that there can only be one moveable object at any given coordinate set.
	
	Running the simulations is build around a basic input update cycle, where input is the player commands, which can be left, right, up, down or a set coordinates. When a command is given the validity of it is checked and if valid the involved objects are then marked to be move in the given direction. Nothing is actually moved in this stage and if a new command is given before and update the previous one is ignored.
	
	The update consists of all objects are moved to their new positions, when they are moved, they are checked for if they should keep moving next update or stop. objects in the bottom of the grid are updated first to ensure items fall correctly.
	
	
	\subsection{GUI}
	A basic GUI using Swing is created, it consists mainly of a visual representation of a given Astro Kid world, and some controls for interacting with this world. the world is visualised using a 2D grid of images which is synchronised with the world using a observer observable pattern.
	
	\subsection{Solving Problems}
	The first thing that happens when trying to solve a problem is a PDDL description is generated from the world. This generated description is a direct representation of the world, so no optimizations are performed on it.

	The generated problem description and a predefined domain description is passed to a new instance Fast Downward as command line parameters. The heuristic is also passed this way, as default the FF heuristic is used as it shown to work best on this domain. The output of Fast Downward is piped to a function that extracts the results of the search. The function extracts all actions with a action cost of 1 (which is the player actions) and translates them to Commands that can be executed on the world. Finally each command is executed on a copy of the world, and error massaged appears if the command isn't valid or no solution is found. 