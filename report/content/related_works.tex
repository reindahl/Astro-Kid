\section{related works}

the two main areas this project revolves around in planning is time and concurrency. 



classic planning normally operates whit action having no duration, and transition between states happening instantaneous.

and this works for many types of problems such as blocks world and sokoban


in reality things .... takes time.... 

the concept of actions having duration isnt anything new,


pddl durative actions....

the concept of temporal planning was introduced in PDDL 2.1, one of the first planners handling this in pddl \cite{durative}

it has the notion of time, but its more directed towards scheduling

preconditions
	start
	under
	end
	
duration

effect
	start
	end
	
	
should end in a safe state




concurency in planning isnt either something new

various adaptation of has been suggest to integrate in a planning system
and in particular MaPL wich builds on





the concept of learning actions schemas has be looked before.....

learning \cite{Action-Schemas} \cite{jacobsen2015a}
\cite{mapl}