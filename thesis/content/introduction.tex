\chapter{Introduction}	
%	planning is an integrated part of it...
%	finding ways of solving problems and doing so efficiently 
	
%	what is planning

	To complete a task efficiently, a plan is often need. Planning and plans are therefore a integrate part of the modern life and appears everywhere. It can be seen in anything from scheduling to route finding. Planning is in it simplest form, having a problem and finding a series of actions that solves the problem (a plan). to easily find plans and ensure the plans actually are efficient planners can be used to automate process.  
	
	%narrow down type of problem....
	Problems can come in many types and shapes, so to simplify the process of automation, planners only works with a subset of the problems types. The particular subset of problems for Astro Kid falls mostly in under classical planning. This type of planning is discrete and works with an atomic state space (see \cite{russell2014a} for more on classical planning).
	
	

%	what is a planner
	There are two general ways of automate process, which is either building a planner for the particular problem or to use a general purpose planner. 
	The general purpose planner has the great advantage that a lot of time and effort can be saved by using already an existing planner, however this often come at the cost of poorer performance.
	To be able to use a planner at all, the problem needs to be described in a way the planner can understand. This trivial for a purpose build planner, for general planners however the problem needs to be described in general way that allow wider span of different types of problems, a planning language. such a language consists of a way of describing initial state, goal state and the possible actions available to manipulate the current state. The most common way of describing this is using the planning language PDDL. 
	

%	in classic planning  the concept of time is missing, actions have no duration and instance effect
%
%	in reality things/actions dont always behave like this, they happen over a time span.... effect under and after, preconditions holds....



	In the ideal world it would be possible to throw a generic planner at any type of problem and having it solve without human intervention. Obvious we are not there yet, and one of the many obstacles on the part, is that the planner needs to know how the domain works before it can solve the problem. This is where learning comes into the picture, by adding learning capabilities to a planner it would become more self-sufficient. Learning in it self can be different things, it can be learning something new or learn to do some thing better. eg. the focus of the learning tracks at IPC, isnt to learn something new but instead optimising what the planners already can do. The focus here is instead on learning something new. %There are different things that can be learnt, what actions are available, which parameters are used, preconditions and effects.



	%Astro kid // purpose
	The project concerns Artificial intelligence used for the flash game Astro Kid\footnote{\url{http://www.agame.com/game/astro-kid}}. This game is a puzzle game, where a Avatar has to move to a designated goal field. In this world the avatar can walk, climb ladders, push objects and use a remote control for stating robots.
	
	The Astro Kid World varies greatly in complexity depending on the size and number of objects, but generally with a slowly growing complexity as it progress trough the levels (due to more and new types of objects being add). One of the challenges with the Astro Kid world is that several actions, have consequences that have a continuous effect over several time steps, and how to represent these effects and interact with them. At the moment no AI exist that can interact with this world. 

	The main goal of this project is to create an agent that can interact and solve problems in Astro Kid world effectively, without prior or only little initial knowledge. The agent should be able to work with a wider array of problems than solely Astro Kid. To achieve this there will first be looked at how far one can get with a generic planner. This means that there also have to be looked at how and how much of the domain can be described in a planning language. 

	Secondly there will be looked at learning how to interact with the environment and the consequences of these interactions. Keeping the knowledge as it progress through the levels is essential to avoid relearning the same things, and therefore it should be able to abstract away from the particular situation and learn in more general terms. The goal of the learning is not to learn the complete domain, but just learn enough to solve the given problems.

%\textbf{Primary goals}
%\begin{itemize}
%	\item Describe what can be described of the domain using PDDL.
%	\item Apply a generic PDDL planner on the domain .
%	\item Creating a more purpose build planner for the domain.
%	\item Implement a system that simulates the Astro Kid world, and allows execution of plans.
%\end{itemize}
%
%\textbf{Secondary goals}
%\begin{itemize}
%	\item Implement direct interaction with the Astro Kid game.
%	\item Learning actions schemas.
%\end{itemize}