\section{Related Works}
The project revolves around planning and learning. The two main areas in planning is effect over time, and concurrency. 

Classic planning normally operates with actions having no duration. This works for many types of problems, such as blocks world, and sokoban. However, in reality things happens over time. The concept of actions having a duration in planning isn't therefore anything new. The concept of temporal planning was introduced in PDDL 2.1 with durative actions. % durative? Det har jeg aldrig hørt før.

% Nope. Amputeret sætning.
One of the first planners handling this in PDDL \cite{durative}. 

% Nope. Concept og notion er for tæt på hinanden til at give ordenlig mening. Går ud fra du mener løsningen fra kilden før?
This concept has the notion of time, but it is directed towards scheduling and problems in general, where the end state of an action can be found from the start.
%
%preconditions
%	start
%	under
%	end
%	
%duration
%
%effect
%	start
%	end
%	
%	
%should end in a safe state



% Nope. Danglish. Du skriver "ej heller" fra dansk, right? F.eks.:
% Concurrency in planning isn't something new, either. Various adaptations....
Concurrency in planning isn't either something new, and various adaptation has been suggest to integrate in a planning system, one of such adaptation is MAPL which builds on multi agent planning\cite{mapl}. The main attempts of using concurrency in PDDL is based around multi agent systems. 

% Ehhhh... Mener du differs from this?
The current problem differentiate since it is a single agent system even though concurrency exists. 

% Dot dot dot?
The concept of learning actions schemas has be looked at multiple times before..... example of different approaches.

% Stryg learning, i "slow learning process".
The concept of learning can involve a vast state space, and therefore it can be a slow learning process. 

% I stedet for "was introduced", så hellere "is described".
A way of speeding up the learning was introduced in \cite{Action-Schemas} with the concept of learning from a teacher. 

% Tror hellere jeg ville skrive "exclude",frem for "not considering"
The idea is to only do things that is known to work and thereby not considering a vast number of possibilities. This approach, however, introduces some limitations such as only allowing positive goals and needing a "teacher" (not self-sufficient). This concept was further extended in \cite{jacobsen2015a}, where the space requirement was reduced and allowing negative preconditions.

A different approach to learning can be seen in \cite{zhuo2010a} which is based around Markov Logic Network and general probability. This approach gives the possibility of learning more complex actions using features such as quantifiers. Unfortunately it does have some limitations, since it does not deal in certainties.
