\section{PDDL Background}
%what is PDDL
The Planning Domain Definition (PDDL) Language is a planning language, created in 1998 by Drew McDermott and the AIPS-98 Planning Competition Committee. It was created to make the international planning competition possible and to breach gab between research and application. The development of PDDL is driven around each IPC, and the development of planners is often also effected by this.
%how does it work
PDDL is an action based language, based on the language STRIPS with some advanced extensions from ADL (Action description language) such as the use of quantifiers. It provides a deterministic, single agent, discrete, and fully observable planning environment. PDDL also works under the closed world assumption, which is that anything not described is false. The current version of PDDL is 3.1.


PDDL is separated into two parts: The problem, and the domain description. The problem description, is the description of a problem instance, which consist of the state of the world and goal condition. The domain description is the rule set of the world, consisting of the actions\footnote{An action consists of Parameters, preconditions, and effects} which is allowed to be performed on the world, and what type of objects is allowed to exist in the world. An example of this can be seen in Appendix \ref{blocks} using the classic blocks domain. This separation means that a single domain description can be paired to many problems, but not necessarily the other way around. For more details on PDDL see \ref{kovacs2011bnf}.
